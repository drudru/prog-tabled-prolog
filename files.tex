\chapter{Handling Large Fact Files}

\section{Indexing}

When a call is made to a predicate that has multiple clauses in its
definition, it is important for efficiency for the system to be able
to quickly determine which defining clauses might match the given
call.  If it can quickly be determined that many clauses can never
match the given clause, then much time can be saved that might
otherwise be spent trying each clause in turn.  This is known as the
clause indexing problem in Prolog implementations.

As a simple example, consider a predicate, say employee/5, that is
defined by a large number, say 10000, of ground facts, which might
come from a table in a relational database.  Then consider a goal:
\begin{verbatim}
| ?- employee(15663,LastName,FirstName,Dept,Salary).
\end{verbatim}
which asks for the name, department, and salary of employee with ID
15663.  A naive implementation of Prolog would look at all 10000
employee facts to determine which one (or ones!) have 15663 in its
first field.  It would do a unification for each of the 10000 clauses,
almost all of which fail.  To avoid doing all this work, all Prolog
implementation support some kind of indexing.  Traditionally Prolog
implementations build an index on the first argument of every
predicate.  That means that they build some kind of structure, usually
a hash table, to allow direct access to clauses given the value of
their first field.  Then when a call is made with a value (and not a
variable) for the first field, the hash table is accessed to find what
clauses might match, and then only for those clauses is full
unification done.  Of course, if a call is made that has a variable in
the first argument, then the hash table (i.e., index) for the first
argument could not be used and all clauses would have to be accessed.
It is easy to see how this could save immense amounts of time in the
evaluation of some queries.  Proper indexing is critical for efficient
XSB query evaluation; it is not at all unusual to be able to speed up
a query by orders of magnitude by finding the best ways to index
predicates defined by a large number of clauses.

XSB supports a variety of ways to index clauses.  The ways supported
differ for predicates whose clauses are compiled (i.e., are stored in
a .xwam file and then loaded into memory) and predicates that are
dynamic (i.e., their clauses are created by asserting them directly
into memory during runtime.)

In XSB compiled predicates can be indexed in only relatively simple
ways.  One can request only one (or no) index be built; the default is
on the first argument, but the programmer can choose any other
argument if desired.  For example, to specify that the compiled
predicate employee/5 is to be indexed on its second argument, add the
following directive to the source file containing the facts for
employee/5:
\begin{verbatim}
:- index(employee/5,2).
\end{verbatim}
The first argument to the index directive indicates the predicate to
be indexed, and the second argument indicates the argument to use to
build the index.  (If the second argument is 0, then no index is
built.)  The index is built using the main function symbol of each
occurrence of the chosen argument.  For example, the default indexing
would be very good for the employee/5 example and query above.  The
first argument of every employee/5 fact is a constant and the index
would take advantage of the constant 15663 in the query.  However,
consider a different situation in which employeeDivID/5 facts had
employee IDs of the form [DivNo,EmpNo], i.e., a list of two subfields
indicating the division the employee was hired into and their ID
number within that division.  In this case all the employeeDivID/5
facts have the same main functor symbol in their first argument, the
list functor, so an index would be completely useless in this
situation, since every first argument main functor symbol would hash
to the same bucket.  So in this case it would be better to store the
two subfields of the ID as separate top-level fields in an employee/6
predicate, and then index on the ID within the division.  This would
give pretty good indexing, but notice that, since we can only index on
one argument, we can't take advantage of knowing both the ID and the
division in a query; the system would have to look through all the
clauses for employees in different divisions that had the same ID
number.

Another limitation of indexing of compiled predicates arises because
we can index on only one argument.  Say we had employee/5 queries
like:
\begin{verbatim}
| ?- employee(15663,LastName,FirstName,Dept,Salary).
and
| ?- employee(ID,'Warren',FirstName,Dept,Salary).
\end{verbatim}
i.e, both queries for data given the employee's ID, and queries for
data given the employee's last name.  Since for compiled clauses we
must choose only one argument for the index, we would have to decide
which form of query is more important, choose that index, and suffer
poor performance for queries of the other form.

The more flexible indexing that XSB supports for dynamic predicates,
those whose clauses are asserted (or dynamically loaded) at runtime,
provides solutions to these problems.  For dynamic predicates,
multiple hash-based indexes, joint indexes, and structure sensitive
indexes are supported.  Alternatively, the programmer can choose to
index a predicate using a new indexing technique, called
trie-indexing.

One can specify a joint index of up to three fields.  For example, we
can create a joint index for (dynamic predicate) employee/6, where the
first two fields are Division and ID, respectively, with the following
directive:
\begin{verbatim}
| ?- index(employee/6,1+2).
\end{verbatim}
In this case an index is built using (the main functor symols of) both
the first and second fields of employee/6 clauses.  Then if a call to
employee/6 provides values for both of those fields, then this index
is used.  Note, however, that if only one of the two fields is
provided in the call, the index (being a hash-index) cannot be used;
so such a call will have to look at {\em all} clauses defining
employee/6.

Indexes for dynamic predicates are built at the time clauses are
asserted into the predicate.  I.e., the index directive just declares
that future asserts into the indexed predicate will have the indicated
index built.  It does {\em not} re-index any clauses currently
defining the predicate.  This means that one (almost) always declares
an index for a dynamic predicate when it has no defining clauses,
before any asserts to that predicate have been done.

In addition to declaring joint indexes, one can declare multiple
indexes for the same dynamic predicate.  For example, we might want
two indexes on employee/5, one on the first argument (the ID) {\em
and} a joint index on the second and third arguments (the last and
first names).  This could be done with with directive:
\begin{verbatim}
| ?- index(employee/5,[1,2+3]).
\end{verbatim}
The programmer provides the multiple index specifications in a list.
So this directive says to build an index on (the main functor symbol
of) the first argument, and to build a joint index on the second and
third argument.  Then if a call to employee/5 provides a value for the
first argument, the first index will be used; if it provides values
for both the second and third arguments, the second (joint) index will
be used.  (If it provides values for all three arguments, the first
index will be used, because it was specified first in the list.)

For dynamic predicates the programmer can also request an argument to
be indexed using ``star indexing'', in which case the index will be a
joint index on (up to) the first five symbols of the indicated
argument.  So, for example, to request a star index the
employeeDivID/5 on its first argument (which, recall, is a two-element
list):
\begin{verbatim}
| ?- index(employeeDivID/5,*(1)).
\end{verbatim}
Now a query such as:
\begin{verbatim}
| ?- employeeDivID([3,1433],Last,First,Dept,Sal).
\end{verbatim}
would use the star index on the first argument, and use both
components of the first argument.  Star indexing can be freely
combined the the default main-functor symbol indexing.  Note that a
query such as:
\begin{verbatim}
| ?- employeeDivID([3,X],Last,First,Dept,Sal).
\end{verbatim}
cannot take advantage of the star index at all, since it doesn't have
values to find a hash value for the joint index.

The programmer can request that a dynamic predicate be trie indexed
by:
\begin{verbatim}
| ?- index(employeeDivID/5,trie).
\end{verbatim}
Trie indexing cannot be combined iwth any other indexing and it
provides a slightly different semantics for clauses defining the
predicate.  First, only fact-defined predicates can be declared as
trie-indexed; no rules are allowed as they are for the other indexing
forms.)  Second, duplicate facts are not supported; i.e., asserting a
duplicate clause into a trie-indexed predicate is a no-op.  Third, the
order of facts is {\em not} preserved; i.e., a fact may be added 






\section{Compiling Fact Files}

(Revise in light of previous (added) section)

Certain applications of XSB require the use of large predicates 
defined exclusively by ground facts.  These can be thought of as
``database'' relations.  Predicates defined by a few hundreds of facts
can simply be compiled and used like all other predicates.  XSB, by
default, indexes all compiled predicates on the first argument, using
the main functor symbol.  This means that a call to a predicate which
is bound on the first argument will quickly select only those facts
that match on that first argument.  This entirely avoids looking at
any clause that doesn't match.  This can have a large effect on
execution times.  For example, assume that p(X,Y) is a predicate
defined by facts and true of all pairs <X,Y> such that $1 <= X <= 20,
1 <= Y <= 20$.  Assume it is compiled (using defaults).  Then the
goal:

\begin{verbatim}
	| ?- p(1,X),p(X,Y).
\end{verbatim}
will make 20 indexed lookups (for the second call to p/2).  The goal
\begin{verbatim}
	| ?- p(1,X),p(Y,X).
\end{verbatim}
will, for each of the 20 values for X, backtrack through all 400
tuples to find the 20 that match.  This is because p/2 by default is
indexed on the first argument, and not the second.  The first query
is, in this case, about 5 times faster than the second, and this
performance difference is entirely due to indexing.

XSB allows the user to declare that the index is to be constructed for
some argument position other than the first.  One can add to the
program file an index declaration.  For example:

\begin{verbatim}
:- index p/2-2.

p(1,1).
p(1,2).
p(1,3).
p(1,4).
...
\end{verbatim}

When this file is compiled, the first line declares that the p/2
predicate should be compiled with its index on the second argument.
Compiled data can be indexed on only one argument (unless a more
sophisticated indexing strategy is chosen.)

\section{Dynamically Loaded Fact Files}

The above strategy of compiling fact-defined predicates works fine for 
relations that aren't too large.  For predicates defined by thousands
of facts, compilation becomes cumbersome (or impossible).  Such
predicates should be dynamically loaded.  This means that the facts
defining them are read from a file and asserted into XSB's program
space.  There are two advantages to dynamically loading a predicate:
1) handling of much larger files, and 2) more flexible indexing.
Assume that the file qdata.P contains 10,000 facts defining a
predicate q(X,Y), true for $1<=X<=100, 1<=Y<=100$.  It could be loaded
with the following command:

\begin{verbatim}
	| ?- load_dyn(qdata).
\end{verbatim}

XSB adds the ``.P'' suffix, and reads the file in, asserting all
clauses found there.  Asserted clauses are by default indexed on the
first argument (just as compiled files are.)

Asserted clauses have more powerful indexing capabilities than do
compiled clauses.  One can ask for them to be indexed on any argument,
just as compiled clauses.  For dynamic clauses, one uses the
executable predicate $index/3$.  The first argument is the predicate
to index; the second is the field argument on which to index, and the
third is the size of hash table to use.  For example,
\begin{verbatim}
	| ?- index(q/2,2,10001).

	yes
	| ?- load_dyn(qdata).
	[./qdata.P dynamically loaded, cpu time used: 22.869 seconds]

	yes
	| ?- 
\end{verbatim}
The index command set it so that the predicate $q/2$ would be indexed
on the second argument, and would use a hash table of size 10,001.
It's generally a good idea to use a hash table size that is an odd
number that is near the expected size of the relation.  Then the next
command, the $load_dyn$, loads in the data file of 10,000 facts, and
indexes them on the second argument.

It is also possible to put the $index$ command in the file itself, so
that it will be used when the file is dynamically loaded.  For
example, in this case the  file woulstart with:
\begin{verbatim}
:- index(q/2,2,10001).

q(1,1).
q(1,2).
q(1,3).
...
\end{verbatim}

Unlike compiled cclauses, asserted clauses can be indexed on more than
one argument.  To index on the second argument if it is bound on
call, or on the first argument if the second is not bound and the
first is, one can use the index command:
\begin{verbatim}
:- index(q/2,[2,1],10001).
\end{verbatim}
This declares that two indexes should be build on $q/2$, and index on
the second argument and an index on the first argument.  If the first
index listed cannot be used (since that argument in a call is not
bound), then the next index will be used.  Any (reasonable) number of
indexes may be specified.  (It should be noted that currently an idex
takes 16 bytes per clause.)





Managing large extensional relations
  load\_dyn, load\_dync, cvt\_canonical.
  Database interface, heterogeneous databases
  (defining views to merge DB's)



\section{Indexing Static Program Clauses}\label{impl-index}

For static (or compiled) user predicates, the compiler accepts a
directive that performs a variant of $unification factoring$ 
\cite{DRRSSSW94}.  

....

\section*{Bibliographic Notes}

The idea of using program transformations as a general method to index
program clauses was presented in a rough form by \cite{HM89}
\cite{DRRSSSW94}  extented these ideas to factor unifications ...
