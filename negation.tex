\chapter{Negation in XSB}

Negation in the context of logic programming has received significant
attention.  Prolog implements a negation-as-failure inference rule,
succeeding the negation of a goal if the goal itself cannot be
finitely successfully proved.  This implements a kind of closed-world
assumption, in that a proposition is assumed to be false if it cannot
be proved to be true.  Recall that with pure Horn clauses, one cannot
prove a negative fact, since the universally true structure (assigning
true to all atoms) is always a model of any set of Horn clauses.  The
closed world assumption allows us to assume facts are false if they
cannot be proved true.  This is a useful operator, which can be used
to represent (and program) interesting situations.

Consider a Datalog database, defining a single relation that
describing the employees of an organization, all defined by simple
ground facts.  Consider the situation in which we ask a simple query:
``Is Bill Gates an employee?''  The query processor will answer
``no''; what does this mean, logically?  It does not mean that it has
proved that not(employee('Bill Gates')) is true in all models of the
database.  The database says only that certain facts are true; it says
nothing about any facts being false.  So a structure in which all the
given employee facts are true and another fact saying Bill Gates is an
employee is also true is a structure satifying all the database facts
and so is a model.  So clearly not(employee('Bill Gates')) is not
logically implied by the database program.  So the ``no'' answer to
the initial query means that we cannot prove that Bill Gates is an
employee; it does not mean that we have proved that Bill Gates is not
an employee.

Consider a simple example of defining the predicate \verb|bachelor|
using the predicates \verb|married| and \verb|male|:

\begin{verbatim}
bachelor(X) :- male(X), \+ married(X).

male(bill).
male(jim).

married(bill).
married(mary).
\end{verbatim}

The rule says that an individual is a bachelor if it is male and it is
not married. (\verb|\+| is the negation-as-failure operator in
Prolog.)  The facts indicate who is married and who is male.  We can
interrogate this program with various queries and we get the following:
\begin{verbatim}
| ?- [negation].
[Compiling ./negation]
[negation compiled, cpu time used: 0.1 seconds]
[negation loaded]

yes
| ?- bachelor(bill).

no
| ?- bachelor(jim).

yes
| ?- bachelor(mary).

no
| ?- bachelor(X).

X = jim;

no
| ?- 
\end{verbatim}
as expected.  The closed-world assumption is applied here: for
example, when there is no fact that says that jim is married, we
assume that he is not married.  Also when there is no fact saying that
mary is male, we assume she is not.

It is important to note that with pure Horn clauses, i.e., simple
implications with conjunctions of (positive) atomic formulas as
antecedents and single atomic formulas as consequents, {\em nothing}
can be proved to be false.  I.e., a set of Horn clauses never
logically implies the negation of any atomic formula.  Given an
arbitrary set of Horn clauses, H, and an arbitrary atomic formula,
A. Consider the structure ST that makes every atomic formula true.  ST
makes H true and makes A true.  Therefore H does not logically imply
not(A).  So it is clear that if we are given a Horn clause program, as
for \verb|married| above, and we ask a negative goal, such as:
\begin{verbatim}
| ?- \+ married(jim).
\end{verbatim}
a ``yes'' answer does {\em not} mean that the program logically
implies \verb|not(married(jim))|.  It means that using these rules
alone, we cannot conclude that married(jim).  

Another way to think about the answers to negative queries to a Horn
clause program is to think in terms of structures.  We saw that every
set of Horn clauses has a unique minimal Herbrand model.  That's the
model constructed by the bottom-up application of the $T_P$ operator.
Given this unique model, we can answer both positive and negative
queries: a positive query is true if it is in the least Herbrand
model; a negative query is true if its subgoal is {\em not} in the
least Herbrand model.  So we can use the unique minimal model to
characterize answers to both positive and negative queries to Horn
programs.

Before we get completely carried away with using negation, we need to
look at situations in which there are problems.  There are two sources
of problems: floundering and nonstratification.  Let's first consider
a floundering query and program.  Say we wrote the bachelor rule as:
\begin{verbatim}
bachelor(X) :- \+ married(X), male(X).
\end{verbatim}
This looks to be an equivalent definition, since after all, the comma
is conjunction so the order of literals in the body of a program (as
simple as this one) shouldn't matter.  But now consider the results of
the same queries as above when they are submitted to this program:
\begin{verbatim}
| ?- bachelor(bill).

no
| ?- bachelor(jim).

yes
| ?- bachelor(mary).

no
| ?- bachelor(X).

no
| ?- 
\end{verbatim}
The answers are fine for the specific queries concerning bill, jim and
mary, but the general query asking for all bachelors fails, whereas we
would expect it to generate the answer jim.  The reason is that the
generated subquery of \verb|\+ married(X)| is able to prove
\verb|married(X)| is true (for X=bill (and X=mary as well)), and so
\verb|\+ married(X)| fails.  The problem is that the
implementation of the operator \verb|\+| only works as we want it to
when it is applied to an atomic formula containing no variables, i.e.,
a ground atomic formula.  It is not able to generate bindings for
variables, but only test whether subgoals succeed or fail.  So to
guarantee reasonable answers to queries to programs containing
negation, the negation operator must be allowed to apply only to
ground literals.  If it is applied to a nonground literal, the program
is said to {\em flounder}.  Prolog systems in general allow the
\verb|\+| operator to be applied to nonground literals and so the
programmer may get unexpected results.  Often another operator,
\verb|not|, is provided which acts just like \verb|\+| except that it
gives an error message when applied to a nonground literal. (In XSB
\verb|not| and \verb|\+| give the same, unsafe, results.)

In fact, when we call the query \verb|\+ married(X)|, with \verb|X| a
variable, it succeeds if it is not the case that there exists a value
for X for which \verb|married(X)| succeeds.  So when a negated goal
contains a variable, the negation has wider scope than the existential
quantifier which binds the variable.  If there are no variables in the
subgoal, then there is no quantifier to interfere.

The other problem that may arise in programs with negation, that of
nonstratification, is a bit more subtle.  While answers to negative
queries to a purely Horn clause program are completely characterized
by the program's least model, in our bachelor example, we are using
negative queries in the definitions of other predicates, not just as
top-level queries.  Such uses can lead to more complex issues.
Consider the following rules:
\begin{verbatim}
shave(john,john).
shave(bill,bill).
shave(barber,X) :- \+ shaves(X,X).
\end{verbatim}
This program says that 'John' shaves himself, and 'Bill' shaves
himself, and that the barber shaves someone if that person does not
shave himself.  The question is: Who shaves the barber?  This is known
as the ``Barber paradox'': If the barber does not shave himself, then
since he shaves all such people, the rule says that he does shave
himself.  And if he does shave himself, then since he only shaves
people who don't shave themselves, he doesn't shave himself.

The question of whether the barber shaves himself can be posed in
Prolog as:
\begin{verbatim}
| ?- shave(barber,barber).
\end{verbatim}
So what should Prolog answer?  Actually Prolog will go into an
infinite loop here, which seems reasonable, since either a yes or no
answer seems incorrect given the argument above.

Consider another example of this kind of phenomenon.  Say we want to
define a predicate, \verb|normal|, that is true of all reasonable
sets.  A set is normal if it doesn't contain itself as a member.  (A
set containing itself is rather weird, when you think about it.  So we
are using the idea of normality to exclude these weird sets.)  To
define \verb|normal|, we give the rule:
\begin{verbatim}
normal(S) :- \+ in(S,S).
\end{verbatim}
where the predicate \verb|in| denotes membership.  Now we want to have
the constant \verb|n| denote the set of all normal sets.  So \verb|X|
is in \verb|n| just in case \verb|X| is a normal set.  The following
rule reflects this:
\begin{verbatim}
in(X,n) :- normal(X).
\end{verbatim}
Now consider what happens if we ask this program whether \verb|n| is
itself a normal set: \verb|normal(n)|, which reduces to 
\verb|\+ in(n,n)|, which reduces to \verb|\+ normal(n)|.  So to show that
\verb|n| is normal, we have to show that \verb|n| is not normal.  This
is essentially a simple formalization of Russell's paradox.

Clearly there is something a little odd with these examples.  The
oddity is that in each case, a predicate, (\verb|normal(n)| or
\verb|shave(barber,barber)|) is defined in terms of its own negation.
Normally we consider rules as defining predicates, and this is an odd
kind of cyclicity which we want to avoid in definitions.  Programs
that avoid such cycles through negation in their definitions are
called {\em stratified} programs.  Notice that Prolog would go into an
infinite loop when asked queries that involve a cycle through
negation.

So we need to have a better idea of what we want these programs that
include negation to mean.  That is, we need a semantics for these
programs.  For positive programs (i.e., Horn clause programs) we have
a very nice and clean semantics; actually we have several of them, all
defining the same thing.  A query to a positive program succeeds if 1)
an instance is logically implied by the program clauses, or 2) if
there is an SLD resolution proof of it, or 3) if an instance is in the
least fixed point of the $T_P$ operator for the program.  The question
is whether we can come up with a simple and clean semantics for
programs with negative goals in their bodies.

\section{Completion Semantics}

The first approach to a semantics of programs with negative literals
in their bodies was proposed by Keith Clark \cite{}, and is known as
Clark's Completion.  The idea here is to use the first idea above for
the semantics: to use the program to come up with a set of formulas
which will logically imply the all the appropriate queries, including
negative ones.  Clearly the implications obtained by treating the
negation operator as logical negation won't do, since they cannot
imply any negative goal.  E.g.
\begin{verbatim}
forall(X,bachelor(X) <- male(X) /\ not(married(X)))

male(bill).
male(jim).

married(bill).
married(mary).
\end{verbatim}
do not imply \verb|not(bachelor(mary))|, which according to our
intuitions we require.  Those formulas cannot imply
\verb|not(bachelor(mary))| since there is a model of these formulas in
which bachelor(mary) is true (one in which everyone is a bachelor,
which satisfies the first rule true.)

So this set of formulas doesn't work, but maybe another one will.
Since we want to think of the first program rule as {\em defining} the
relation bachelor, and we usually use ``if and only if'' for
definitions, we might turn all the ``if''s into ``if and only if''s as
follows:
\begin{verbatim}
forall(X,bachelor(X) <-> (male(X) /\ not(married(X))))

forall(X,male(X) <-> X=bill \/ X=jim)

forall(X,married(X) <-> X=bill \/ X=mary)
\end{verbatim}
Its easy to see that for the first rule, we simply made the quantifier
explicit and turned the ``if'' into an ``iff''.  For the male/1
predicate, we had to create a single rule for all the facts, creating
a rule with a single ``if'', and then we changed that ``if'' into
``iff''.  The intuitive meaning of the second rule that defines the
relation male/1 is that X is a male if and only if X is either bill or
jim.

Now this is a set of logical formulas, and we would like the meaning
of a ground query to be true if this set of formulas (called the
completion of the program) logically implies the query.  There is one
more issue we need to handle to make this work.  Given just the
if-and-only-if rules above, we still cannot prove that bill is not a
bachelor.  The problem is that it may be that both constants bill and
jim refer to the same individual in the model.  (Intuitively this
might be more understandable had I chosen to name the second male as
james instead of as bill.)  So to be able prove that bill is not
married, we have to know that \verb|not(bill=jim)|.  So to make the
completion semantics work, we add additional inequality axioms to say
that no two constants are equal.  We also need to add inequality
axioms to say that functions are freely interpreted, i.e., that two
function applications are equal only if they are the same function
symbol applied to equal arguments.  

With these additional inequality axioms and the completion of the
bachelor program, we can prove not(bachelor(bill)) as follows:

From \verb|forall(X,bachelor(X) <-> (male(X) /\ not(married(X))))|, we
can infer 
\verb|(bachelor(bill) <-> (male(bill) /\ not(married(bill))))|.  
And from this and \verb|married(bill)| and
\verb|not(bill=jim)|, we can infer \verb|not(bachelor(bill))|.  So for
this program we have succeeded in giving a meaning to programs with
negation that satisfy our intuitions.

However, there are other programs for which the completion gives
somewhat unintuitive results.  For example, consider the following
transitive closure program:
\begin{verbatim}
e(a,a).
e(b,a).

p(X,Y) :- e(X,Y).
p(X,Y) :- e(X,Z),p(Z,Y).
\end{verbatim}
Its completion is (assuming free variables are universally
quantified):
\begin{verbatim}
e(X,Y) <-> (X=a /\ Y=a) \/ (X=b /\ Y=a)

p(X,Y) <-> e(X,Y) \/ exists(X,e(X,Z) /\ p(Z,Y))
\end{verbatim}
Here we again put the two standard rules for transitive closure
together into a single rule and then turned the ``if'' into and
``iff''.

Now consider the transitive closure of this simple graph; it is
\{p(a,a), p(b,a)\}, since there are no paths other than the length-one
paths corresponding to the edges.  So since p(a,b) is actually not in
the transitive closure, we would like to have the completion of this
program {\em logically imply} that is {\em not} in the transitive
closure.  I.e., we would like the completion to imply
\verb|not(p(a,b))|.  But it doesn't.

Notice that the structure that makes the following atoms true: 
\begin{verbatim}
  {e(a,a),e(b,a),p(a,a),p(b,a),p(a,b),p(b,b)}
\end{verbatim}
satisfies all the rules.  It clearly satisfies the first rule.  To see
that is satisfies the second rule, consider all that rule's instances:
\begin{verbatim}
1. p(a,a) <-> e(a,a) \/ exists(Z,e(a,Z) /\ p(Z,a))
2. p(b,a) <-> e(b,a) \/ exists(Z,e(b,Z) /\ p(Z,a))
3. p(a,b) <-> e(a,b) \/ exists(Z,e(a,Z) /\ p(Z,b))
4. p(b,b) <-> e(b,b) \/ exists(Z,e(b,Z) /\ p(Z,b))
\end{verbatim}
Clearly the first and second instances are true since both sides
evaluate to true.  Consider the third instance: the left-hand side
(p(a,b)) is true, e(a,b) is false, but let Z be a, and then we see
that e(a,a) /\ p(a,b) is true in the structure, so the right-hand side
is also true, making this entire thrid formula true.  And consider the
fourth instance: the left-hand side is true, e(b,b) is false, but if
we take Z to be a, we get (e(b,a) /\ p(a,b)) which is true in the
structure; so both sides are true and the fourth instance is true as
well.  So all the instances of the completion of the second rule of
the program are true, so there exists a model of the completion (this
one) that makes p(a,b) true.  So clearly the completion does not
logically imply not(p(a,b)).  So what this means is that under the
completion semantics our definition of ``transitive closure'' does not
define real transitive closure.

For this reason, many have concluded that the completion semantics is
not the desired semantics for logic programs.  It may handle programs
with negation in a reasonable way, but it goes a step backward for
positive programs, in particular one as simple as transitive closure.

The problem here can be seen to boil down to rules like:
\begin{verbatim}
p :- p.
\end{verbatim}
In the least model of this program, p is false.  But the completion of
this rule: \verb|p <-> p| is a tautology so it does not imply that p
is false (or that p is true); it imposes no constraint on \verb|p| at
all.  By looking carefully at our transitive closure example, we can
see this ``self-supporting'' phenomenon going on, with p(a,b), for
example.

Recall what we wanted the program completion to do for us for this
program.  We wanted to construct a set of first-order formulas derived
from the transitive closure program that would logically imply exactly
the facts in the transitive closure and the negation of facts not in
the transitive closure.  It turns out that this is not possible; there
is {\em no} such set of first-order formulas.  This is a basic theorem
of first-order logic. \cite{}.

So rather than basing our semantics of logic programs with negation on
the notion of logical consequence of a first-order theory, let's turn
to look at a possible least-fixed point characterization.

\section{Negation through Fixed Points}

Let's consider how we might understand the bachelor program using
least fixed points.  Again, the program:
\begin{verbatim}
bachelor(X) :- male(X), \+ married(X).

male(bill).  male(jim).

married(bill).  married(mary).
\end{verbatim}
Here we can first compute the least fixed point of the program
consisting just of the two married facts, which (trivially) contains
just those two facts.  Now having completing that fixed point, we can
compute a fixed point for bachelor, which depends on the negation of
married.  




The paradoxical examples show that definitions of predicates that
depend on their own negations are problemmatic.  So our first approach
to providing a semantics for negation in logic programming will
involve prohibiting such programs.  A program that doesn't include
cyclic definitions that involve negation is called {\em stratified}.

A Prolog program is {\em Predicate Stratified} if each predicate, P,
can be assigned an integer (a stratum), S(P), such that for every pair
of predicates $P_1$ and $P_2$ in rule in the program of the form:
\begin{verbatim}
P_1(..) :- ..., P_2(...), ...
\end{verbatim}
\verb|S(P_1) >= S(P_2)|, and for every pair of predicates $P_1$ and
$P_2$ in rule in the program of the form:
\begin{verbatim}
P_1(..) :- ..., \+ P_2(...), ...
\end{verbatim}
\verb|S(P_1) > S(P_2)|.

For example, consider the bachelor program: 
\begin{verbatim}
bachelor(X) :- male(X), \+ married(X).

male(bill).
male(jim).

married(bill).
married(mary).
\end{verbatim}
This program has three predicates: \verb|bachelor|, \verb|male|, and
\verb|married|.  We can assign strata as follows:
\begin{verbatim}
S(male) = 1
S(married) = 1
S(bachelor) = 2
\end{verbatim}
and this stratification function satisfies the requirements that come
from the one rule, i.e., that S(bachelor) \verb|>=| S(male), and
S(bachelor) \verb|>| S(married).  So this program is said to be
predicate stratified.  There are many possible stratification
functions that satisfy the necessary constraints.  We can define a
``least'' one, in which all the strata are as small (positive)
integers as possible.  The one given here is least.

Consider the program that defines normal sets:
\begin{verbatim}
in(X,n) :- normal(X).
normal(S) :- \+ in(S,S).
\end{verbatim}
For this program S(in) must be \verb|>=| S(normal), because of the
first rule, and S(normal) must be \verb|>| S(in), because of the
second rule.  It is clear that this is not possible, so no
stratification function exists for this program, and the program is
said to be not predicate stratified, or unstratified, or
unstratifiable.

There is a simple algorithm for determining whether a stratification
function exists for a program and finding one if there does.
The algorithm tries to construct a stratification function but will
fail if none exists.  The algorithm is as follows: First assign every
predicate the stratum 1.  Then for every pair of predicates $P_1$ and
$P_2$ that appear in a rule of the program of the form:
\begin{verbatim}
P_1(..) :- ..., \+ P_2(...), ...
\end{verbatim}
If $S(P_1)$ \verb|<=| $S(P_2)$, then set $S(P_1)$ to be $S(P_2)+1$.
Repeat this until no change needs to be made, in which case the final
$S$ is a stratification (and a least one), or until for some predicate
P, S(P) \verb|>| $n$, where $n$ is the number of distinct predicates
in the program, in which case there is no stratification.

Given a stratification for a program, we can define a unique Herbrand
model for the program bottom-up using the idea the $T_P$ operator.  We
define the meanings of the predicates in a stratified program in the
order of their stratification numbers.  First look at all the
predicates at stratum 1.  None of these depend negatively on any
predicate, so those rules, denoted by ${P1}$, are a set of Horn
clauses.  We can construct the least model of these rules, using the
$T_{P1}$ operator for the stratum 1 program, which generates the set of
ground atomic formulas, $M_{P1}$, that are logical consequences of
${P1}$.

Now $M_{P1}$ defines all the predicates at stratum 1.  Now consider all
predicates at stratum 2, ${P2}$.  They depend on other stratum 2
predicates and stratum 1 predicates, perhaps negatively on stratum 1
predicates.  We can now apply the $T_{P2}$ operator iteratively to
$M_{P1}$ and now interpreting a negative atomic formula %not~A$in the
body of a rule in ${P2}$ as true iff $A$ is not in $M_{P1}$.  We
continue in this way through the strata, at each step using the atoms
computed at an earlier step to interpret the negative literals in the
bodies of rules.  In this way we can construct in iterated fixed
point, which will be a model of rules in the program.  This model is
called the {\em Perfect Model} of the stratified program.

For example:
\begin{verbatim}
1. p :- q, \+r, s.
2. q :- p.
3. q :- s.
4. r :- \+s, t.
5. s.
\end{verbatim}
A stratification can be computed as follows:
\begin{verbatim}
S(p) = 1
S(q) = 1
S(r) = 1
S(s) = 1
S(t) = 1

S(p) = 2, by rule 1, and S(r) = 1
S(q) = 2, by rule 2, and S(p) = 2
S(r) = 2, by rule 4, and S(s) = 1

S(p) = 3, by rule 1, and S(r) = 2
S(q) = 3, by rule 2, and S(p) = 3
\end{verbatim}
So this program is stratified, and its stratification is:
\begin{verbatim}
S(p) = 3
S(q) = 3
S(r) = 2
S(s) = 1
S(t) = 1
\end{verbatim}
With this stratification we can compute the perfect model.
${P1}$:
\begin{verbatim}
5. s.
\end{verbatim}
and $M_{P1}$: \verb|{s}|.  Then ${P2}$ is
\begin{verbatim}
4. r :- \+s, t.
\end{verbatim}
and $M_{P2}$ = \verb|{s}|, since \verb|\+s| is false (since s is in $M_{P1}$)
and then %P_3% is:
\begin{verbatim}
1. p :- q, \+r, s.
2. q :- p.
3. q :- s.
\end{verbatim}
and $M_{P3}$ is \verb|{s, q, p}|, since s is in $M_{P2}$ and r is not in
$M_{P2}$. 

There are a few things to note about the perfect model.  It is an
iterated fixed point, and so is defined by transfinite induction.
I.e., we may have to iterate to infinity at one stratum before we can
move to the next stratum.  So this means that the perfect model may
not be recursively enumerable.  

When a program is predicate stratified, it is generally agreed that
the perfect model provides the correct semantics for it.  However,
there are many programs that are not predicate stratified but for
which Prolog computes what seems to be the correct answers.  Consider
for example the definition of even natural numbers (represented in
successor notation):
\begin{verbatim}
even(0).
even(succ(X)) :- \+ even(X).
\end{verbatim}
Prolog computes this correctly.  But the program is not predicate
stratified, since the predicate \verb|even/1| depends on itself
negatively.  But notice that when it does, its argument is smaller, so
there actually is no cycle through negation in the definition; the
definition ``grounds out'' at 0.  Also note that we can transform any
predicate stratified propositional program into a nonstratified
(predicate logic) program simply by introducing one unary predicate,
say, c/1, and then replacing every propositional symbol, P, in the
original program by c(P).  Clearly the program hasn't changed in any
fundamental sense, but if the original program had any negation in it
at all, the transformed program will not be predicate stratified; the
only predicate is c/1 so it will depend negatively on itself. 

To address these problems, more refined definitions of stratification
were developed.  In particular the notion of ``local stratification''
was defined.  The idea of local stratification is to ground the
program, and then to determine if the (usually infinite) set of ground
atomic formulas can be placed in strata so that there is no cycle
through negation.  For example, the even/1 program above is locally
stratified, since we can assign the stratum 1 to the ground atomic formula
even(0), and n+1 to the formula even(succ(...succ(0))) with n
successors.  With this stratification, it's easy to see that any atom
depends negatively only on atoms with smaller strata.

However, there are disadvantages to local stratification.  For example
it is undecidable whether a finite predicate logic program is locally
stratified, which means that were we to use that as the definition of
a meaningful program, it would undecidable whether a program has
meaning.

So the search continued for a definition that would give meaning to
all programs, stratified (in whatever form) or not.


[How/whether to include this following...]

As an example of stratified negation, consider the situation in which
we have a set of terms and a nondeterministic reduction operation over
them.  Then given a term, we want to reduce it until further
operations don't simplify it any more.  We will allow there to be
cycles in the reduction operation and assume that terms that reduce to
each other are equivalently fully reduced.

This situation can be abstractly modeled by considering the terms to
be nodes of a directed graph with an edge from N1 to N2 if the term at
N1 directly reduces to the term at N2.  Now consider the strongly
connected components (SCCs) of this graph, i.e. two nodes are in the
same SCC if each can be reached from the other.  We will call an SCC a
final SCC if the only nodes reachable from nodes in that SCC are
others in that SCC.  Now given a node, we want to find all nodes
reachable from that node that are in final SCCs.

So first we define reachable:
\begin{verbatim}
:- table reachable/2.
reachable(X,Y) :- reduce(X,Y).
reachable(X,Y) :- reachable(X,Z), reduce(Z,Y).
\end{verbatim}
Next we can define \verb|reducible| to be true of nodes that can be
further reduced, i.e., those nodes from which we can reach other nodes
that cannot reach back:
\begin{verbatim}
reducible(X) :- reachable(X,Y), tnot(reachable(Y,X)).
\end{verbatim}
\verb|tnot| is the negation operator for tabled goals.  It checks to 
see that the call doesn't flounder, giving an error message if it
does.  It can be applied only to a single goal, and that goal must be
a tabled predicate.  With this predicate we can next define the
predicate \verb|fullyReduce| that is true of pairs of nodes such that
the first can be reduced to the second and the second is not further
reducible:
\begin{verbatim}
:- table reducible/1.
fullyReduce(X,Y) :- reachable(X,Y),tnot(reducible(Y)).
\end{verbatim}
Note that we must table \verb|reducible| because \verb|tnot| can only
be applied to predicates that are tabled.

So with these definitions and the following graph for reduce:
\begin{verbatim}
reduce(a,b).
reduce(b,c).
reduce(c,d).
reduce(d,e).
reduce(e,c).
reduce(a,f).
reduce(f,g).
reduce(g,f).
reduce(g,k).
reduce(f,h).
reduce(h,i).
reduce(i,h).
\end{verbatim}
we can ask queries such as:
\begin{verbatim}
| ?- fullyReduce(a,X).

X = c;

X = h;

X = d;

X = k;

X = i;

X = e;

no
| ?- 
\end{verbatim}
which returns all nodes in final SCCs reachable from node \verb|a|.

However, we may now wish to generate just one representative from each
final SCC, say the smallest.  We can do that with the following
program:
\begin{verbatim}
fullyReduceRep(X,Y) :- fullyReduce(X,Y), tnot(smallerequiv(Y)).

smallerequiv(X) :- reachable(X,Y), Y@<X, reachable(Y,X).
\end{verbatim}

Now we get:
\begin{verbatim}
| ?- fullyReduceRep(a,X).

X = c;

X = h;

X = k;

no
| ?- 
\end{verbatim}

Note that this is an example of a stratified program.  The predicate
\verb|reachable| is in the lowest stratum; then \verb|reducible| is
defined in terms of the negation of \verb|reachable| so it is in the
next stratum; then \verb|fullyReduce| is defined in terms of the
negation of \verb|reducible|, so it is in the third stratum.
\verb|smallerequiv| is in the first stratum with \verb|reachable|;
and \verb|fullyReduceRep| is in the same stratum as
\verb|fullyReduce|.


\section{General Negation}

There have been many proposals for a semantics for general logical
programs, i.e., programs with unrestricted negation.  When considering
general semantics, there are a few very simple (propositional)
programs that we can use to show their particular characteristics.
The programs are the following:
\begin{verbatim}
p :- p.
\end{verbatim}
This is a simple program in which p depends (only) on itself.  Notice
that Prolog goes into an infinite loop on this program, and Tabled
Prolog terminates with p being false.

The second program is a direct negative loop.
\begin{verbatim}
p :- \+ p.
\end{verbatim}

The final program has two propositional symbols, each of which depends
directly on the other negatively.
\begin{verbatim}
p :- \+ q.
q :- \+ p.
\end{verbatim}

We will use these programs to help understand (and distinguish) the
various semantics for general logic programs that we consider in the
following sections.

\subsection{Clark's Completion Semantics}

Idea is to turn the implications of rules into biconditionals and then
to treat closed world negation as logical negation. 

So take program:
Put rules with same head together into one rule (with disjuctive
bodies).  Add existential quantifiers for variables in the body but
not the head.

Then turn the if's into iff's.

This is called the completion of the program.

Example:

even(0).
even(s(X)) :- \verb|\+|even(X).

Completion:

\begin{verbatim}
even(X) <==> exists(Y, (X = 0 \/ (X=s(Y) /\ not(even(Y)))).
\end{verbatim}

Also must add ``Clark's Equality Theory'' (CET), to define equality:
not(a=b) for every pair of constants a and b.  
f(X1,..,Xn) = g(Y1,..,Yn) iff f=g, and Xi=Yi for all Y.  
not(X=f(..,T,..)) if X appears in term T.

Then the meaning of a program is the set of logical consequences of
the completion of the program and CET.

So is even(s(s(0))) a logical consequence of the completion of the
program and CET?

\begin{verbatim}
even(s(s(0))) <==> exists(Y,(s(s(0))=0\/(s(s(0))=s(Y)/\not(even(Y)))))
    <==> not(even(s(0)))
    <==> not(exists(Y,(s(0)=0\/(s(0)=s(Y)/\not(even(Y))))))
    <==> not(not(even(0)))
    <==> not(not(exists(Y,(0=0\/...))))
    <==> not(not(true))
    <==> true.
\end{verbatim}

Yes.  So Clark's completion gives us what we want for the even
program.

What does it do with:
\begin{verbatim}
p :- p.
\end{verbatim}
The completion of this program is simply \verb|p<==>p|, which is a
tautology.  So its logical consequences are just the tautologies, so
it does not imply either p or not(p).  So from this program we cannot
conclude that p is true or that p is false.  Notice that this differs
from the perfect model semantics in which we can conclude from this
program that p is false.

This does have an interesting effect on the transitive closure
definition.  Consider the program:
\begin{verbatim}
e(a,a).
e(b,a).

p(X,Y) :- e(X,Y).
p(X,Y) :- e(X,Z),p(Z,Y).
\end{verbatim}
We can ground this program into:
\begin{verbatim}
e(a,a).
e(b,a).

p(a,a) :- e(a,a).
p(a,b) :- e(a,b).
p(b,a) :- e(b,a).
p(b,b) :- e(b,b).
p(a,a) :- e(a,a),p(a,a).
p(a,a) :- e(a,b),p(b,a).
p(a,b) :- e(a,a),p(a,b).
p(a,b) :- e(a,b),p(b,b).
p(b,a) :- e(b,a),p(a,a).
p(b,a) :- e(b,b),p(b,a).
p(b,b) :- e(b,a),p(a,b).
p(b,b) :- e(b,b),p(b,b).
\end{verbatim}
And now the completion is:
\begin{verbatim}
e(X,Y) <==> (X=a,Y=a) ; (X=b,Y=a).

p(a,a) <==> e(a,a) ; (e(a,a),p(a,a)) ; (e(a,b),p(b,a)).
p(a,b) <==> e(a,b) ; (e(a,a),p(a,b)) ; (e(a,b),p(b,b)).
p(b,a) <==> e(b,a) ; (e(b,a),p(a,a)) ; (e(b,b),p(b,a)).
p(b,b) <==> e(b,b) ; (e(b,a),p(a,b)) ; (e(b,b),p(b,b)).
\end{verbatim}
Now using the first biconditional in the 2nd and 4th rule, we can
simplify this to:
\begin{verbatim}
e(a,a).
e(b,a).
not(e(a,b)).
not(e(b,b)).

p(a,a).
p(a,b) <==> p(a,b)
p(b,a).
p(b,b) <==> p(a,b)
\end{verbatim}
And now note that just as in the positive loop, we cannot prove either
that p(a,b) is true or that it is false.  So this means that the
completion semantics does not imply that in this graph, you cannot get
from a to b.  So this program does not define transitive closure.

So this shows that the completion semantics does not agree with the
perfect model semantics.  In the perfect model semantics, p(a,b) is
definitely false.  For this reason, the completion semantics is not
normally considered a good semantics for logic programs.  People feel
that the above transitive closure program ought to indeed define
transitive closure.  And since under the completion semantics, it
doesn't, the completion semantics is not a good semantics.

It is RE, and it is the logical consequences of some set of formulas...


\subsection{Stable Model Semantics}

Another approach to giving semantics to general logic program was
proposed by Michael Gelfond and Vladimir Lifschitz \cite{hjk} and is
known as Stable Models.  This approach defines a model for a program
as a fixed point of a particular operator, the Gelfond-Lifschitz (or
GL) operator (not surprisingly.)

We will consider propositional programs.  Stable models of predicate
programs are defined in terms of the (possibly infinite) ground
instantiation of the program.  A model is represented as a set of
proposition symbols, the model making all symbols in the set true and
all those not in the set false.  For any general propositional logic
program, the GL operator takes a set of propositional symbols and
returns a set of proposition symbols.  It is defined using a
transformation of the given program called the GL-reduct of a program.
Given a program P and a set of proposition symbols S, the GL-reduct of
P is the program obtained from P by 
\begin{enumerate}
\item
deleting any rule which has a negative literal in its body whose
proposition symbol is in S, and
\item
deleting from the body of any rule any negative literal whose
proposition symbol is not in S.
\end{enumerate}
For example, the GL-reduct of program:
\begin{verbatim}
p :- q, \+r, s.
q :- \+s.
r := \+q, s.
\end{verbatim}
with respect to the set of symbols:
\begin{verbatim}
{p, q}
\end{verbatim}
is
\begin{verbatim}
p :- q, s.
q.
\end{verbatim}
One can think of the GL-reduct of a program with respect to a set S as
being the program one would get by assuming the propositions in S are
true when interpreting negative body literals of program rules.

Notice that the GL-reduct of any program with respect to any set S is
a definite program.  In the process of constructing the GL-reduct we
delete all negative literals in a program.

Now we can define a Stable Model of a general propositional logic
program.  A stable model of a program P is a set of proposition
symbols M such that the least model of the GL-reduct of P with respect
to M is M.  I.e., it is a fixed point of the operator obtained by
taking the least model of the GL-reduct.

Proposition: If a program P is predicate stratified, then P has
exactly one stable model, which is the perfect model.

Examples:  For the program:
\begin{verbatim}
p :- p.
\end{verbatim}
the only stable model makes \verb|p| false.  This follows from the
previous proposition, as well as from direct construction.

For the program:
\begin{verbatim}
p :- \+p.
\end{verbatim}
there are {\em no} stable models.

For the program: 
\begin{verbatim}
p :- \+q.
q :- \+p.
\end{verbatim}
there are two stable models: one that makes p true and q false, and
one that makes p false and q true.

Stable models have several drawbacks.  Computing the stable model of a
propositional program is NP-complete, so the best general algorithm
known is exponential.  Also, it does not have the ``relevance''
property.  That is, to find whether there is a stable model that makes
a particular proposition symbol p true, one has to look at the entire
program, not just the propositions that are used to define p.  For
example, if the program contains a rule \verb|p :- \+p|, then there is
no stable model at all, regardless of what the rest of the program is
(as long as it doesn't involve p.)


\subsection{Well-Founded Semantics}
Another approach to defining the meaning of logic programs with
default negation is the Well-Founded Semantics, which uses a 3-valued
logic \cite{vangelder-et-al}.  In a 3-valued logic a proposition may
be true or false or {\em undefined}.  By using the truth value
undefined judiciously, we can provide a single 3-valued model for each
logic program with negation.

We will motivate the definition of the well-founded semantics by first
considering what we can infer if we have only partial knowledge of
some predicates that are used, but not defined, in an old-fashioned
definite logic program.

Consider a situation in which we have some (definite) rules that
define some predicates inductively, and use other predicates in their
definitions, but don't define them.  We will call the used, but not
defined, predicates as open predicates.  We will assume that we have
partial knowledge of the truth values of atoms of the open predicates.
As a simple, concrete example, consider our friend {\em reachability}
in a graph:
\begin{verbatim}
reach(X,Y) :- edge(X,Y).
reach(X,Y) :- reach(X,Z), edge(Z,Y).
\end{verbatim}
Here \verb|reach/2| is defined using the \verb|edge/2| predicate, but
\verb|edge/2| is not defined.  We can say that the \verb|reach|
definition is parameterized by the open predicate \verb|edge/2|; a
definition of the \verb|edge/2| predicate determines a complete
definition of the \verb|reach/2| predicate.

But perhaps we have only partial knowledge of the \verb|edge/2|
relation: say we know that there is an edge from \verb|a| to \verb|b|,
an edge from \verb|b| to \verb|a|, no edge from \verb|a| to \verb|c|
and no edge from \verb|b| to \verb|c|, but we don't know about other
possible edges.  (We'll assume the graph has only these three nodes.)
The question is: What can we conclude about the \verb|reach/2|
relation using this incomplete information about the open predicate
\verb|edge/2|?  Intuitively, we know that \verb|b| is reachable from
\verb|a|, \verb|a| is reachable from \verb|b|, \verb|a| is reachable
from \verb|a|, \verb|b| is reachable from \verb|b|, \verb|c| is not
reachable from \verb|a|, and \verb|c| is not reachable from \verb|b|.
But we don't know if \verb|a| is reachable from \verb|c|, if \verb|b|
is reachable from \verb|c|, or if \verb|c| is reachable from \verb|c|.
Let's be more precise (and general) concerning how we might come to
such conclusions.

We assume that the known true \verb|edge| facts are T =
\{\verb|edge(a,b)|, \verb|edge(b,a)|\} and the known false \verb|edge|
atoms are F = \{\verb|edge(a,c)|, \verb|edge(b,c)|\}.  How do we
conclude what \verb|reach| atoms can be known true and what
\verb|reach| atoms can be known false?  To find ones that must be
true, we can just add the known facts to the (definite) program and
find its least model.  Anything true in that model must be true even
if some (or all) of the unknown open atoms would turn out to be false.
For our example, we get the following program:
\begin{verbatim}
edge(a,b).
edge(b,a).
reach(X,Y) :- edge(X,Y).
reach(X,Y) :- reach(X,Z), edge(Z,Y).
\end{verbatim}
Now we see what is implied by this set of rules: here we get
\{\verb|reach(a,b)|, \verb|reach(b,a)|, \verb|reach(a,a)|,
\verb|reach(b,b)|\} by taking the least fixed point of this program.
These are facts that must be true in any model consistent with the
partial knowledge we have of the \verb|edge/2| relation.  We have made
conservative assumptions and what is still true under these most
conservative assumptions must indeed be true.  We can say that this
program defines the ``definitely true'' atoms for \verb|reach|.

We have just seen how to conclude what \verb|reach| atoms must be
true; now how do we determine what atoms must be false?  We similarly
create a program using our assumptions of what \verb|edge| atoms are
true and false, but this time we assume that all \verb|edge| atoms
{\em not} known to be false are actually true.  So the program we get
is:
\begin{verbatim}
edge(a,b).
edge(b,a).
edge(c,b).
edge(c,a).
edge(a,a).
edge(b,b).
edge(c,c).
reach(X,Y) :- edge(X,Y).
reach(X,Y) :- reach(X,Z), edge(Z,Y).
\end{verbatim}
The seven \verb|edge| facts are those not known to be false,
i.e. those that might be true.  We use this program, taking the least
model, to see what \verb|reach| atoms ``might be true''.  We can say
this program defines the ``possibly true'' atoms.  In this case, we
get: \{\verb|reach(a,b)|, \verb|reach(b,a)|, \verb|reach(c,b)|,
\verb|reach(c,a)|, \verb|reach(a,a)|, \verb|reach(b,b)|,
\verb|reach(c,c)|\}.  Any \verb|reach| atom {\em not} in this set will
have to be false in any model consistent with our initial \verb|edge|
assumptions.  So we conclude that any \verb|reach| atom in the
complement of this set must be false in any consistent model.  So we
know that \verb|reach(b,c)| and \verb|reach(a,c)| must be false.  We
have made liberal assumptions about what might be true and anything
still not true under these most liberal assumptions must indeed be
false.

To review, we have described a way to use partial or incomplete
knowledge of facts that are used in the definitions of inductively
defined predicates to infer information about those predicates,
information including what instances must definitely be true and what
are possibly true (and whose complement must definitely be false.)
The idea for determining the definitely true defined instances is to
assume that all defining facts not known true are false.  Then
anything that must be true in this situation is definitely true.  To
determine the definitely false atoms, we assume all defining atoms not
known to be false are indeed true; anything that still isn't possibly
true under these assumptions must definitely be false.

We can use this idea to give a semantics to programs with negative
literals in their bodies; i.e., inductively defined predicates that
use negations in their definitions.  For simplicity we will consider
only propositional programs here.  In order to use a negative literal
to help infer a fact, we must know that the literal is true, i.e., the
atom that is negated is false.  But how can we determine such things?
We will use the idea we have just developed of reasoning with partial
information about open predicates to approach this problem.  We start
by initially disconnecting the negative literals in the bodies of
clauses from their positive forms and just thinking of them as new
propositions.  E.g., for the literal \verb|tnot(p)|\footnote{Recall
  that tnot is the tabled negation operator in XSB.}, we introduce a
new proposition symbol, say \verb|neg_p|.  We then replace all
negative literals in the program with their new positive forms to get
a purely positive program with open predicates.  The resulting program
can be seen as similar to the old program but now parameterized by the
newly introduced \verb|neg_*| predicates.

So let's apply our approach to reasoning about definitions with
partial knowledge about open predicates.  We begin by assuming that we
know nothing about these new open predicates, i.e., none of their
atoms are known true or false.  So to determine what defined
predicates must definitely be true, we interpret all the \verb|neg_*|
predicates as false, and see what is true in the resulting least
model.  Those defined atoms are now known true.  And to find the false
atoms, we interpret all the open predicates as true, take the least
model to see what could conceivably be true, and then take its
complement to find the definitely false atoms.

Let's consider an example; the initial general program with negation
is:
\begin{verbatim}
p :- r, tnot(t).
q :- r, tnot(s), tnot(u).
r :- s.
r.
s :- tnot(q), r.
t.
\end{verbatim}
We transform it to its open form:
\begin{verbatim}
p :- r, neg_t.
q :- r, neg_s, neg_u.
r :- s.
r.
s :- neg_q, r.
t.
\end{verbatim}
To determine the well-founded model of the original program, we will
maintain two programs with the open propositions: one that tells us
what atoms must be true (called the {\em definitely true} program),
and a program that tells us what atoms could possibly be true (called
the {\em possibly true} program) and therefore tells us what atoms
must be false.  Each program will be a conservative approximation and
we will iteratively modify the programs to improve their accuracy.

The initial definitely true program and definitely false program are:
\begin{verbatim}
%  definitely true               %  possibly true

p :- r, neg_t.                   p :- r, neg_t.
q :- r, neg_s, neg_u.            q :- r, neg_s, neg_u.
r :- s.                          r :- s.
r.                               r.
s :- neg_q, r.                   s :- neg_q, r.
t.                               t.
                                 neg_p.  neg_q.  neg_r. neg_s.  neg_t.  neg_u.
\end{verbatim}
In the definitely true program we assume that none of the \verb|neg_*|
propositions are true; in the possibly true program we assume they are
all true.  If we take the least model of the definitely true program,
we get \{\verb|r|, \verb|t|\}.  These propositions will true
regardless of the truth values of the \verb|neg_*| propositions, so we
will want them to be true in the well-founded model of the original
program.  And the least model of the possibly true program (on the
defined propositions) is \{\verb|p|, \verb|q|, \verb|r|, \verb|s|\}.
So any defined proposition symbol {\em not} in this set must be false,
and we will want it false in the well-founded model.

Now we can try to use each of these programs to improve the accuracy
of the other.  There really is a connection between the pair of
propositions, say \verb|p| and \verb|neg_p|: if one is true, then the
other should be false.  So if we deduce that \verb|r| is definitely
true in the well-founded model, then we know that \verb|neg_r| is
definitely false, i.e., not possibly true.  And that allows us to
update our possibly true program by deleting the fact for
\verb|neg_r|.  Similarly, if we know that \verb|u| is definitely false
then \verb|neg_u| is definitely true, and we can update our definitely
true program by adding \verb|neg_u| to it.  Having changed our
definitely true and possibly true programs, we can again find their
least fixed points and see if we have learned something new that will
allow us to further update the programs.  We continue to add
\verb|neg_*| atoms to the definitely true program and remove
\verb|neg_*| atoms from the possibly true program in this way until we
learn nothing new.  The resulting programs define the well-founded
semantics of the original program.

Consider this process for our example; the least model of the true
program contains r and t, so these are known true.  Since they are
known true, \verb|neg_r| and \verb|neg_t| must be false, so we can
remove them from the possibly true program, improving our estimate of
the possibly true atoms and getting an updated possibly true program:
\begin{verbatim}
%  definitely true               % possibly true

p :- r, neg_t.                   p :- r, neg_t.
q :- r, neg_s, neg_u.            q :- r, neg_s, neg_u.
r :- s.                          r :- s.
r.                               r.
s :- neg_q, r.                   s :- neg_q, r.
t.                               t.
                                 neg_p.  neg_q.  neg_s.  neg_u.
\end{verbatim}
Now looking at the least model of this new possibly true program, we
see that neither u nor p is possibly true, i.e., not in the least
model of the possibly true program and thus must be false.  So we can
improve our estimate of the true atoms by adding \verb|neg_u| and
\verb|neg_p| to our definitely true program, obtaining:
\begin{verbatim}
%  definitely true               % possibly true

p :- r, neg_t.                   p :- r, neg_t.
q :- r, neg_s, neg_u.            q :- r, neg_s, neg_u.
r :- s.                          r :- s.
r.                               r.
s :- neg_q, r.                   s :- neg_q, r.
t.                               t.
neg_p. neg_u.                    neg_p.  neg_q.  neg_s.  neg_u.
\end{verbatim}
Now looking at the current versions of the two programs: the neg
version of every defined atom in the least model of the definitely
true program has been removed from the possibly true program; and the
neg version of every defined atom not in the least model of the
possibly true program has been added to the definitely true program.
So we can no longer improve our estimates of the definitely true and
possibly true atoms, which leaves us with the well-founded model of
the original program.  The atoms true in the well-founded model are
the defined atoms in the least model of the final definitely true
program; the atoms false in the well-founded model are the defined
atoms {\em not} in the least model of the final possibly true program.
Thus for this program \verb|r| and \verb|t| are true in the
well-founded model, \verb|p| and \verb|u| are false, and \verb|s| and
\verb|q| are undefined.

To review this process: the definitely true program starts with no
\verb|neg_*| atoms and gains them as their positive counterparts are
found not to be possibly true.  The possibly true program starts with
all the \verb|neg_*| atoms and loses them as their positive
counterparts are found to be definitely true.  The process continues
until no more improvements can be made.  Then we read off the
well-founded model from the final programs.



\section{Other Stuff to think about?}

We need examples! Planning? Use for-all type problems, e.g. to find if
all nodes reachable from a given node are red, find if it is not the
case that there exists a node reachable from the given node that is
not red.


*********************************

Issues of safety.


\section{Approximate Reasoning}

Use course prerequisites example. Introduce undefined truth value.  Add
undetermined facts for courses currently being taken.  Then requests for 
whether have satisfied requirements will give:
	true if satisfied without any current course
	false if not satisfied even if all current courses are passed
	undetermined if satisfaction depends on outcome of current courses.



Categorization examples?  Propositional Horn clauses for bird
identification.  Allow negation as failure, explicit negation.

\begin{verbatim}
cardinal :- red, crested.
bluejay :- blue, crested.
robin :- red_breasted, ~crested.

Use undef as a kind of null value?:
binarize relation
undef :- ~undef.
%emp(Name,Sal,Age) -> empsal/2 and empage/2
emp(david,50000,_|_) is represented as ``facts'':

empsal(david,50000).
empage(david,_X) :- undef.

or empage(david,X) :- between(45,X,55),undef.
(so can fail if not between 45 and 55.)
\end{verbatim}


\section{Representation of Partial Knowledge with Well-Founded Models}

We can use the undefined truth value to encode the fact that we don't
know whether a particular proposition is true or false.  As an
example, we will use a (simplification of a) simple game called {\em
Wumpus}.  The game is played on a grid, and a hunter starts on some
grid square and moves across the board searching for the one other
square that contains the pot of gold.  The hunter can move left,
right, up, or down, and his goal is to move from where he is to the
square that has the gold and pick it up.  However, there are some
squares on the grid that are dangerous.  Some squares contain pits,
and if the hunter moves to such a square, he falls into the pit and is
killed.  The hunter can sense when he is adjacent to a pit by feeling
a breeze.  So if he moves to a square and feels a breeze, he knows
that some adjacent (immediately left, right, up, or down) square
contains a pit.  So the hunter must move from where he is to the gold
while avoiding falling into any pit.

The idea will be to represent the state of knowledge of the hunter by
a well-founded model.  The following are fixed rules of how the game
works:

\begin{verbatim}
%% grid provides the size of the grid.
grid(4,4).

%% A neighbor is a square to either side or above or below the current
%%  square (and still on the grid.)
neighbor(I,J,NI,J) :- NI is I+1, grid(MI,_), NI =< MI.
neighbor(I,J,NI,J) :- NI is I-1, NI >= 1.
neighbor(I,J,I,NJ) :- NJ is J+1, grid(_,MJ), NJ =< MJ.
neighbor(I,J,I,NJ) :- NJ is J-1, NJ >= 1.

%% feel(I,J,Sense) if the hunter has felt "Sense" on square I,J

:- table feel/3.
%% the hunter can feel only one of breeze and no_breeze
feel(I,J,breeze) :- tnot(feel(I,J,no_breeze)).
feel(I,J,no_breeze) :- tnot(feel(I,J,breeze)).

:- table safe/2.
%% A square is safe if it is not a pit
safe(I,J) :- tnot(pit(I,J)).
%% and if the hunter has sensed something there it is also safe.
safe(I,J) :- visited(I,J).
%% A square is safe if the hunter feels no breeze on some neighbor.
safe(I,J) :- 
	neighbor(I,J,NI,NJ),
	visited(NI,NJ),
	feel(NI,NJ,no_breeze).

:- table pit/2.
%% A square is a pit if it is not safe
pit(I,J) :- tnot(safe(I,J)).
%% A square is a pit if some neighbor feels a breeze but no other of
%% its neighbors is a pit.
pit(I,J) :-
	neighbor(I,J,NI,NJ),
	feel(NI,NJ,breeze),
	tnot(another_neighbor_unsafe(NI,NJ,I,J)).

%% Another neighbor (not (NI,NJ) is a pit.
:- table another_neighbor_unsafe/4.
another_neighbor_unsafe(I,J,NI,NJ) :-
	neighbor(I,J,PI,PJ),
	(PI =\= NI ; PJ =\= NJ),
	tnot(safe(PI,PJ)).
\end{verbatim}

Consider the following state, in which the hunter has visited squares
1,1 and 1,2, and felt a breeze on 1,1 and no breeze on 1,2:

\begin{verbatim}
%% a sample state
feel(1,1,breeze).
feel(1,2,no_breeze).
visited(1,1).
visited(1,2).
\end{verbatim}

With these rules and this state, we compute the following:

\begin{verbatim}
| ?- consult(wumpus,[spec_off]).
[Compiling .\wumpus]
[wumpus compiled, cpu time used: 0.1100 seconds]
[wumpus loaded]

yes
| ?- pit(2,1).

yes
| ?- safe(2,2).

yes
| ?- safe(2,3).

undefined
| ?- 
\end{verbatim}

We see that the hunter can infer that there is a pit in square 2,1,
since it is the neighbor of a cell in which a breeze has been felt
(1,1), and all other neighbors of that cell (only 1,2) is known to be
safe.

So with this representation, we can develop a strategy for the hunter
to safely move around the grid as follows.  The hunter uses WFM of the
current knowledge base to determine what adjacent cells are safe and
moves to one of them.  He then adds to the knowledge base that he has
visited the new square and what he sensed in that square.  And he
repeats the process.

If the new information was undefined in the previous state, we can
update the previous WFM in XSB by using \verb|force_answer_true/1|.
However, in the example above, the hunter needs to add the
\verb|visited| fact for the new square he moved to, and this fact was
false in the previous WFM, not undefined.  I.e., we need to change the
knowledge of visited from false to true, not from undefined to true.
So in this system the WFM must be computed anew in each new state.
(There may be another way to encode this problem so that all knowledge
acquisition would be moving from facts with undefined truth values to
those with distinct values, but I don't see it now.)

Now in an actual game, a situation may arise in which the hunter has
several places he can move but is unable to infer that any of them is
safe.  In this case, if he wants to find the gold, he must make a
potentially dangerous move to a square which may be a pit.  The WFM
provides no information as to whether some square might more likely be
safe than another square.  We could extend the WFM with probabilities
to try to infer which squares are more likely to be safe.  This can be
done within the PRISM (Sato) framework...


leftover stuff for now...........




XSB, with its tabling, does not improve over Prolog in handling
floundering queries; all calls to negative subgoals must be ground in
XSB for the closed-world interpretation of negation to be computed.
XSB does extend Prolog in allowing nonstratified programs to be
evaluated, and we will discuss how it does that later in the chapter.
However, first we will explore how we can use tabling with stratified
programs to compute some interesting results.
*****end of leftover............

There have been a number of approaches to defining the meaning of
logic programs with the negation-as-failure operator.  We will
describe one general approach that will shed light on several others.
The intuitive idea is that Prolog programs are to be understood as
rules that define predicates, not as first-order logical formulas.
When we are just dealing with positive Horn rules, then these two
intuitions coincide in some sense.  We saw that the least model of a
Horn program defines the set of predicates whose atoms occur in heads
of rules in the program, and the facts for those predicates are
exactly the logical implications of the Horn clauses.  So we can use
logical implication to define relations from Horn clauses.  However,
when we include the negation operator, and want to interpret it in a
``negation as failure'' way, things are not so simple.

(Aside and Overview)

The story for the following sections on proposed semantics for Prolog
with \verb|\+|.

1. We want to ``characterize'' the negation operator (\verb|\+|) in
Prolog.  What does it precisely mean?  Prolog goals have 3 outcomes:
yes, no, and infinite loop.  So our characterization of the meaning of
goals will need to have 3 values: true, false and undefined (or
undetermined, or unknown.)  There are several generally accepted
approaches to providing such a semantics.

2. The first is the completion semantics.  Here a goal is
true if it is logically implied by the ''completion'' of the program;
it is false if its negation is logically implied by the completion,
and undefined otherwise.  Result, a) doesn't exactly characterize
prolog with \verb|\+|: consider goal p and rule p :- p,f.  Prolog
gives undefined (infinite loop); completion gives false. and b)
completion (and Prolog) give undefined for p:-p, and thus doesn't give
transitive closure as meaning of the program for transitive closure.
So this tells us that maybe we don't want a semantics for \verb|\+| in
Prolog, but for \verb|\+| in Tabled Prolog.

3. Define Perfect models, only for predicate stratified programs.
Gives nice definition of \verb|\+| in Tabled Prolog, but only for stratified
programs.  All goals are true or false (no undefined.)  Correctly
characterizes all terminating Tabled Prolog programs (when \verb|\+| is
implemented in the ``obvious'' way.)

4. Define Stable Model semantics for all (inc nonstratified) programs.
Program has 0 or more stable models: goal is true under SM semantics
if true in all SMs, false if false in all SMs, and undefined
otherwise.  Problems: exponentiality, nonrelevance.  And for many
interesting uses (e.g., combinatorial problems) SM characterizes a
solution, e.g. a coloring of a graph.  So interesting aspect of an SM
is the truth of various goals in the same model (i.e., a coloring)
rather than the truth of a goal in all models (e.g., node 1 must be
green in all valid colorings.)

5. Define Well-Founded semantics.  ...

6. Partial Stable Models, and connections between SM and WFS.

(END Aside and Overview)

(Take 2 Aside and Overview)

1. How to give a semantics for Normal Logic Programs: Prolog programs
with unconstrained negation.

2. Define stratified programs, and give their semantics.

3. Define semantics of parameterized programs: i.e., Some predicates
given by relations, and take least fixed point using those definitions
for those predicates.

4. Define WFS as the limit of an iteration of parameterized programs:
For every predicate symbol p, create a symbol p'.  Replace in original
program every p under a negation by p'. Then start by taking meaning
of this modified program parameterized by p'={} and p'=Universal. And
alternating and iterating...

5. Then define FO(ID): 
a. A set of defined predicates, and a set of open predicates
b. A set of rule with atoms of defined predicates in their heads
	(and open or defined, pos or neg, literals in their bodies)
c. A set of first-order sentences (using defined and open predicates)

A model of an FO(ID) program is any first-order structure (for the
defined and open predicates), such that the relations for the defined
predicates are the (2-valued) well-founded model parameterized by the
relations of the open predicates, and all the FO-sentences are true in
the structure.

6. Then Stable Models can be defined in FO(ID).




1. Given a program (with negation).  If we know *nothing* about the
truth or falsity of the negative subgoals, what could we still infer
about true and false propositions?  

Goals are true if they can be proved with absolutely NO help from
the negative literals.  So delete all rules with negative literals and
take least model, and all propositions true there must be true.

Goals are true if they can not be proved no matter what help they got
from the negative literals.  So delete all negative literals from the
program rules, take least model, and propositions not in it are false.

2. So given a program with negations, we use the above idea to
determine known true propositions and known false propositions, T1 and
F1.  Now given that we know the truth values of these subgoals, we
reduce the program using them.  I.e., Delete each rule with a negative
literal whose proposition is in T1, and delete from each remaining
rule each negative literal whose proposition is in F1.  Now this
program (along with T1 and F1) should give us the same meaning as the
original program.  Now repeat the process of step 1 using this
program, to get more propositions known true and known false.

3. And continue until there is no change in the true and false sets.
The resulting T and F sets determine the well-founded model, with
propositions in neither T nor F being undefined.

(END COMMENT)

