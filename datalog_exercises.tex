\section{Exercises}

\begin{enumerate}

\item
In this problem you will explore how XSB handles transitive closure
for binary relations defining various graphs.  The first task is to
define predicates that generate those graphs.  In each case, we will
assume that the graph predicate will always be called with the first
argument bound.  Each graph will be parameterized by a number M, which
will be defined by a Prolog fact: size/1.  So if M is 10000, there
will be a fact:
size(10000).
defined in your Prolog program.

\begin{enumerate}

\item
Define a binary predicate cycle/2 such that cycle(i,i+1) for 
\begin{verbatim}
0 =< i < M, and cycle(M,0).
\end{verbatim}
Draw a picture of this graph (with integers as nodes) for M = 8.

\item
Define a binary predicate tree/2 such that 
\begin{verbatim}
tree(i,2*i+1) for 0 =< i < M, and
tree(i,2*i+2) for 0 =< i < M.
\end{verbatim}
Draw a picture of this graph (with integers as nodes) for M = 4.

\item
Define a binary predicate dline/2 such that
\begin{verbatim}
dline(i,i+1) if 0 =< i < M and i mod 4 = 0  (i.e. i is a multiple of 4)
dline(i,i+2) if 0 =< i < M and i mod 4 = 0
dline(i+1,i+4) if 0 =< i < M and i mod 4 = 0
dline(i+2,i+4) if 0 =< i < M and i mod 4 = 0
\end{verbatim}
Draw a picture of this graph (with integers as nodes) for M = 12.
\end{enumerate}

\item
For each of the predicates define in problem 1, give the rules for
transitive closure of that graph, possible in three ways: the
right-recursive definition, the left-recursive definition, and the
doubly-recursive definition.  They should be named:

\begin{verbatim}
tc_cycle_rr/2 (for the right recursive transitive closure of cycle)
tc_cycle_lr/2 (for the left recursive..)
tc_cycle_dr/2 (for the doubly recursive..)

tc_tree_rr/2  (right recursive tree)
tc_tree_lr/2  (left recursive tree)

tc_dline_rr/2 (right recursive)
tc_dline_lr/2 (left recursive)
\end{verbatim}

You are to populate the following benchmarking table:

\begin{verbatim}
graph          tabled    M1   M2   M3  M4  M5  M6  M7  M8  M9  M10
------------------------------------------------------------------
tc_cycle_rr/2    Y 
tc_cycle_lr/2    Y
tc_cycle_dr/2    Y

tc_tree_rr/2     N

tc_tree_rr/2     Y
tc_tree_lr/2     Y

tc_dline_rr/2    N

tc_dline_rr/2    Y
tc_dline_lr/2    Y
\end{verbatim}
where M1 - M10 are integers, in increasing order that are chosen
to show reasonable ranges of cpu times for all the benchmarks.  These
benchmarks can vary significantly in speed, so to get a reasonable
time for the fastest you'll need a reasonably large M10; any time
greater than 30 seconds should be represented as TO (for time out).
(I.e., cancel any query that takes longer, using ctrl-c.)  The
"tabled" column having "Y" means that the corresponding transitive
closure definition is declared as tabled.  If it is "N", it is not
tabled, but simple Prolog is used.

The values in the cells are the cpu-times for running XSB on the
0-source query for the row.  E.g., the number in the <tc\_cycle\_rr/2,
M1> cell should be the cputime to run the query tc\_cycle\_rr(0,X) (when
tc\_cycle\_rr/2 is tabled) to find all X's reachable from 0.

This can be done by the following query:
\begin{verbatim}
| ?- cputime(T0),(tc_cycle_rr(0,_),fail ; true), cputime(T1), 
  Time is T1-T0, writeln(Time).
\end{verbatim}

You might find the following predicate helpful:

\begin{verbatim}
:- dynamic size/1.           % needed at beginning of code!
:- import for/3 from basics.
test(Goal,M) :-
    N is M//10,              % 10 equal intervals
    for(I,1,10),             % for I=0 to 10
    NM is I*N,               % the next decile
    abolish_all_tables,      % clear all tables
    retractall(size(_)),     % clear old size
    assert(size(NM)),        % set new size for base predicate
    cputime(T0a),            % find cputime from start
    (call(Goal),fail;true),  % do the query, all answers
    cputime(T1a),
    Timea is T1a-T0a,
    writeln([Goal,NM,Timea]), % write out goal, size, and cputime
    fail.
\end{verbatim}

And example call:
\begin{verbatim}
| ?- test(tc_cycle_lr(0,_),1000000).
\end{verbatim}
\end{enumerate}
