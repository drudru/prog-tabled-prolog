\section{Exercises}

\begin{enumerate}

\item (Is this exercise appropriate here?  If I write a chapter on Abstract
  Interpretation, maybe move there and add an exercise on AI.).

Write a XSB program to parse a simple
programming procedural language and construct an abstract syntax tree
for it.  The syntax is taken from the Modula-3 programming language,
but greatly simplified.

The grammar is following. (The empty string is indicated by e, for
epsilon; | indicates alternative rules (or), and parentheses indicate
grouping except when quoted in which case they indicate tokens):

\begin{verbatim}
module ::= MODULE identifier ; block identifier .
block ::= declaration_sequence block_body
block_body ::= END | BEGIN statement_sequence END
declaration_sequence ::= declaration_sequence declaration | e
declaration ::= VAR variable_declaration_list ;
variable_declaration_list ::= variable_declaration 
         (e | ; variable_declaration_list)
variable_declaration ::= identifier_list : type
identifier_list ::= identifier (e | , identifier_list)
statement_sequence ::= statement statement_sequence | e
statement ::= assignment_statement ; | if_statement ; |
              while_statement ; | print_statement ;
assignment_statement ::= identifier := expression
if_statement ::= IF expression THEN statement_sequence
                   ELSE statement_sequence END
while_statement ::= WHILE expression DO statement_sequence END
print_statement ::= PRINT '(' expression ')'
type ::= INT
expression ::= simple_expression relop simple_expression | simple_expression
relop ::= < | > | =
simple_expression ::= simple_expression addop term | term
addop ::= + | -
term ::= term mulop primary | primary
mulop ::= * | /
primary ::= identifier | integer | '(' simple_expression ')'
\end{verbatim}

You will be provided a scanner in file scanner.P, which will read
a source file containing a program, and produce a set of token facts
that can be processed by your DCG grammar.  The scanner will return
keywords directly as tokens; identifiers as terms ident(IdentName);
integers as int(IntegerVal); and special characters as themselves as
single tokens.  The tokens will be asserted as word/3 facts, and you
can use your DCG to process them, as indicated in parse\_file/2 below.

Your grammar must recognize all programs according to this grammar
and, for each recognized program, produce an abstract syntax tree
(AST) for that program.  The AST must have the following form:

\begin{enumerate}

\item Expressions should have operators as interior nodes (named by the
operator symbol) and have as children the ASTs for their operands.
Identifiers should be ident(Identifier) and integers int(Integer).

\item An assignment statement AST should have the form
assign(Var,ExprAST); an if-then-else statement AST should have the
form if(ExprAST,ThenStmtAST,ElseStmtAST); a while statement AST should
have the form while(ExprAST,DoStmtAST); a print statement AST should
have the form print(ExprAST).

\item A statement sequence AST should be a list of statement ASTs.

\item An identifier list AST should be a list of identifiers.

\item A variable declaration is an IdentifierListAST, as is a variable
declaration list; as is a declaration; as is a declaration sequence.
(Since the langauge has only INT variables, type indications are
unnecessary; we need only the names of the variables.)

\item A block AST is of the form block(IdentifierListAST,StmtListAST).

\item A module AST is of the form module(Name,BlockAST).
\end{enumerate}
You should write a predicate parse\_file/2, which takes the name of a
file containing a program, and if the program is in the language of
the grammar, returns its Abstract Syntax Tree.

You may import the scanner by putting the scanner.P file in your
directory and then adding the following declaration to your grammar
program:
\begin{verbatim}
:- import scan_file/2 from scanner.
\end{verbatim}

Then you can call it with the following code:

\begin{verbatim}
parse_file(ProgramFile,AST) :-
    scan_file(ProgramFile,Length),
    module(AST,0,Length).
\end{verbatim}

\end{enumerate}
