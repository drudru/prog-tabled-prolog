\chapter{Programming with Sets}

\section{Introduction}
Prolog is fundamentally based on sets of tuples of terms, i.e.,
relations.  But sometimes it is useful to manipulate explicitly sets
of terms, as values of Prolog variables.  For small sets, one often
uses sorted lists (without duplicates) as a canonical representation,
but since this representation has linear lookup and linear update, it
is unsuited for handling large sets.  XSB supports a powerful abstract
data type for manipulating large sets of ground terms.

The set representation has essentially log time lookup and update and
is canonical, and by use of the XSB facility of ``interned terms'',
can efficiently be combined with tabling.  It turns out to be
surprisingly powerful.

\section{Interned (or HashCons-ed) Terms}

Describe interning, and predicate {\tt intern\_termhash/2}.

\section{The prolog\_db Data Structure}

Describe this library.  Use much from the documentation of
lib/prolog\_db.tex in the XSB system distribution.

\section{Conclusion (or Summary?)}
if any...
