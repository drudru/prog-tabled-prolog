\chapter{Managing Large, Updatable State}

\section{Introduction}
Prolog (and XSB) supports the dynamic updating of its global database,
i.e., its set of clauses that define the predicates that it uses.  For
a predicate to be updatable, it must be declared to be {\em dynamic}.
Changes to the global clase set (a.k.a. database) are persistent
across backtracking.  Any Prolog program that uses dynamic predicates
is inherently nondeclarative, and thus to be avoided when possible.
The question is when is it reasonably possible.  This chapter
describes how we can often use the {\tt prolog\_db} set data
structure, described in the previous chapter, along with methodologies
for using it to avoid the need for assert and retract, the Prolog
builtins to change dynamic predicates in the global clause set.

\section{The prolog\_db Data Structure}

Describe this library.  Use much from the documentation of
lib/prolog\_db.tex in the XSB system distribution.

\section{Examples}

Various examples...

\section{Conclusion (or Summary?)}
if any...
