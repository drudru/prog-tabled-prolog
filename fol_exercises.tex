\section{Exercises}

\begin{enumerate}
\item
Prolog has infix operations that allow you to write propositional
expressions, such as \verb|(p /\ q \/ not(p /\ q)|).  We will use this syntax
to represent propositional expressions, with \verb|/\| meaning "and", \verb|\/|
meaning "or" and not meaning "not".

We will represent a truth assignment as a list, such as:
\begin{verbatim}
[p=true,q=false,r=false]
\end{verbatim}
where each element in the list defines the mapping of a propositional
symbol to true or false.

Write a Prolog predicate eval\_tv(PropExp,TruthAsgn,TV) which takes a
propositional expression, PropExp, and a truth assignment in
TruthAsgn, and returns the truth value of that expression under that
truth assignment in TV.

For example 
\begin{verbatim}
| ?- eval_tv((p /\ q \/ not(q \/ p)),[p=false,q=false],TV).

TV = true ;

no
| ?- 
\end{verbatim}

\item
Extra Credit: (Not for the squeamish.  Only for those with time and
interest.)  Write a predicate that, given a propositional formula,
will print out its truth table.

Prolog has builtin predicates write(Term) that writes the term Term,
and nl, which writes out a newline.  Try for example:
\begin{verbatim}
| ?- write(hello),nl,write(world).
\end{verbatim}

\item
Write a DCG to recognize and parse propositional sentences.

\item
Write a Prolog predicate prop\_clause/2 that takes a propositional
formula (using \verb|/\|, \verb|\/|, and \verb|not|, as in the
previous problems and produces an equivalent list of clauses.  A
clause is represented as a list of literals.  This predicate should in
effect convert the propositional formula into conjunctive normal form
and then produce the resulting list of clauses.  (You do not have to
create the CNF form if you find a more direct way to get the same
result.)

\item
Write a Prolog predicate resolve/1 that takes a list of clauses,
such as those generated in problem 1, applies the resolution strategy,
succeeding if the clauses are unsatisfiable, and failing otherwise.

One way to write resolve is as a recursive predicate that chooses a
pair of resolvable clauses, generates the resolvent, adds it to the
set of clauses if it's new, and recurses, until the empty clause is
generated, or no new clauses can be generated.  Now you need to choose
only one resolvable pair at each step; you don't want to backtrack and
choose all pairs nondeterministically.  To do this in Prolog, you can
use a metapredicate, once/1, which when applied to a predicate call,
will return only the first answer the call generates.

So for example, we get:
\begin{verbatim}
| ?- once(member(X,[a,b,c])).
X = a;
no
| ?- 
\end{verbatim}

Using once/1, you can choose only one of the many ways resolvants
might be generated at each step.

\begin{enumerate}
\item
Add writeln/1 calls at appropriate places in your code so that you can
see, for each the resolvant, what clauses were used to generate it.
\item
Test your program with reasonable clause sets.  Come up with an
interesting example with some semantics for the propositions, and
explain what your proof system does.
\end{enumerate}

\item
This problem involves representing a first-order structure in a Prolog
program and evaluating the meaning of formulas in such a structure.

\begin{enumerate}
\item
Consider the representation of a first-order structure in Prolog.  For
our problem, the structure will be an abstraction of a simple world
with several people in it and several relations relating the people.
We use the following representation:

\begin{enumerate}

\item
To represent the domain, we will use a unary Prolog predicate,
dom/1, which will return each element of the domain.
So, for example, the fact:
\begin{verbatim}
dom(1).
\end{verbatim}
would indicate that 1 is in the domain of the structure.

\item
To represent the relations, we will use a binary predicate,
relation/2, which returns pairs in which the first is a predicate
symbol (of arity n) in the language, and the second is a list of
length n representing a tuple in the corresponding relation.
So, for example, the fact:
\begin{verbatim}
relation(<,[3,17]).
\end{verbatim}
would indicate that the predicate symbol \verb|<| is mapped to a relation
that contains the ordered pair \verb|<3,17>|.

\item
To represent the functions, we will use a binary predicate,
function/2, which returns pairs in which the first is a function
symbol (of arity n) in the language, and the second is a list of
length n+1 representing the function arguments and its value for those
arguments in the graph of the corresponding function.
So, for example, the fact:
\begin{verbatim}
function(+,[2,3,5]).
\end{verbatim}
would indicate that \verb|+| is a function symbol in the language that is
mapped by the structure to a binary function that, given arguments 2
and 3, returns 5.
\end{enumerate}

Task: Give definitions of these three Prolog predicates for a
first-order structure representing the following situation:

\begin{enumerate}
\item
There are 4 people: john\_jones, mary\_smith, bill\_rogers,
jane\_doe.

\item
john\_jones and bill\_rogers are men, where men are represented by a
unary relation symbol 'man'.  mary\_smith and jane\_doe are women, where
women are represented by a unary relation symbol 'woman'.

\item
Some people love other people, represented by a binary relation
symbol 'loves'.  In particular john\_jones loves mary\_smith and she
loves him back.  And bill\_rogers loves jane\_doe, but jane\_doe loves
john\_jones.

\item
Every person has a spouse, represented by a unary function symbol
'spouse'.  In particular, john\_jones and mary\_smith are married (and
thus are each other's spouses) and bill\_rogers and jane\_doe are also
married.
\end{enumerate}

\item
Write first-order formulas (in the language above) to represent the
meanings of the following English sentences.  (You may use implication
(\verb|->|) as a logical connective, if you wish.)  The constants we
have in the logic are 'John', 'Mary', 'Jane', and 'Bill' (which refer
to the obvious objects in the structure), the unary predicate symbols
are 'man' and 'woman', the binary predicate symbol 'loves', the unary
function symbol 'spouse'.

\begin{enumerate}
\item
John loves Mary.
\item
Bill loves Jane but Jane loves John.
\item
Mary loves a man.
\item
Jane does not love her spouse.
\item
Every man loves his spouse.
\item
Not every woman loves her spouse.
\item
Every man loves a woman.  (i.e., for every man there is a woman that he loves)
\item
There is a man that every woman loves.
\end{enumerate}

\item
We will represent first-order formulas in Prolog as shown in the
following examples.

\begin{verbatim}
forall(x1,exists(y2,loves(x1,y2)))
forall(x1,not(man(x1))\/exists(y,woman(y)/\loves(x1,y)))
forall(x,loves(x,spouse(x)))
\end{verbatim}

So variables are represented by Prolog atoms that begin with the letters
x, y, or z followed by a (possibly empty) sequence of digits.

Universal quantification is represented by the Prolog term with forall
at the root and its left child a variable, and its right child a
formula.  And existential quantification is similar.

The representation of terms and the other form of formulas should be
clear from the examples.

Task: Write a Prolog predicate: 
\begin{verbatim}
mean_fmla(+Formula,+VariableAsgn,-TV)
\end{verbatim}
that takes a first-order formula, a variable assignment, and returns
true if the formula is true in the structure represented by the Prolog
predicates dom/1, relation/2 and function/2.  A variable assignment is
represented as a list of terms Varable=Object.  For example, a
variable assignment for the structure of problem 1 might look like:
[x=john\_jones,y1=jane\_doe].  We will assume that VariableAsgn has
terms for every variable that appears free in Formula.

We will need an auxiliary predicate, call it mean\_term, where
mean\_term(+Term,+VariableAsgn,-Obj) takes a first-order logical term,
a variable assignment, and returns the object that term means in the
structure defined by dom/1 and friends.

We will need a Prolog predicate to break apart a term that is an
atomic formula (or a function application) into its components.  We
use the builtin (infix) predicate =.. as the following example shows:

\begin{verbatim}
| ?- loves(john,spouse(x2)) =.. Term.
Term = [loves,john,spouse(x2)];
no
| ?- 
\end{verbatim}

So given a left-hand argument of a term (tree), =.. breaks it apart
into a list, where the first element of the list is the atom at the
root of the term tree, and the tail of the list is a list of the terms
that are the immediate children of the root.  =.. works in either
direction, i.e., it will also construct a term tree from a list
representing the root atom followed by its children (but we don't need
that direction in this problem.)

Task: Define the necessary predicates in Prolog, run mean\_fmla/3 on
all the example formulas you came up with in problem 2 just above.
Make up some other formulas, including some that are false in the
structure of problem 1, and run mean\_fmla on them, explaining why they
get the answers that they do.
\end{enumerate}

\end{enumerate}
