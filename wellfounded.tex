\subsection{Well-Founded Semantics}
Another approach to defining the meaning of logic programs with
default negation is the Well-Founded Semantics, which uses a 3-valued
logic \cite{vangelder-et-al}.  In a 3-valued logic a proposition may
be true or false or {\em undefined}.  By using the truth value
undefined judiciously, we can provide a single 3-valued model for each
logic program with negation.

We will motivate the definition of the well-founded semantics by first
considering what we can infer if we have only partial knowledge of
some predicates that are used, but not defined, in an old-fashioned
definite logic program.

Consider a situation in which we have some (definite) rules that
define some predicates inductively, and use other predicates in their
definitions, but don't define them.  We will call the used, but not
defined, predicates as open predicates.  We will assume that we have
partial knowledge of the truth values of atoms of the open predicates.
As a simple, concrete example, consider our friend {\em reachability}
in a graph:
\begin{verbatim}
reach(X,Y) :- edge(X,Y).
reach(X,Y) :- reach(X,Z), edge(Z,Y).
\end{verbatim}
Here \verb|reach/2| is defined using the \verb|edge/2| predicate, but
\verb|edge/2| is not defined.  We can say that the \verb|reach|
definition is parameterized by the open predicate \verb|edge/2|; a
definition of the \verb|edge/2| predicate determines a complete
definition of the \verb|reach/2| predicate.

But perhaps we have only partial knowledge of the \verb|edge/2|
relation: say we know that there is an edge from \verb|a| to \verb|b|,
an edge from \verb|b| to \verb|a|, no edge from \verb|a| to \verb|c|
and no edge from \verb|b| to \verb|c|, but we don't know about other
possible edges.  (We'll assume the graph has only these three nodes.)
The question is: What can we conclude about the \verb|reach/2|
relation using this incomplete information about the open predicate
\verb|edge/2|?  Intuitively, we know that \verb|b| is reachable from
\verb|a|, \verb|a| is reachable from \verb|b|, \verb|a| is reachable
from \verb|a|, \verb|b| is reachable from \verb|b|, \verb|c| is not
reachable from \verb|a|, and \verb|c| is not reachable from \verb|b|.
But we don't know if \verb|a| is reachable from \verb|c|, if \verb|b|
is reachable from \verb|c|, or if \verb|c| is reachable from \verb|c|.
Let's be more precise (and general) concerning how we might come to
such conclusions.

We assume that the known true \verb|edge| facts are T =
\{\verb|edge(a,b)|, \verb|edge(b,a)|\} and the known false \verb|edge|
atoms are F = \{\verb|edge(a,c)|, \verb|edge(b,c)|\}.  How do we
conclude what \verb|reach| atoms can be known true and what
\verb|reach| atoms can be known false?  To find ones that must be
true, we can just add the known facts to the (definite) program and
find its least model.  Anything true in that model must be true even
if some (or all) of the unknown open atoms would turn out to be false.
For our example, we get the following program:
\begin{verbatim}
edge(a,b).
edge(b,a).
reach(X,Y) :- edge(X,Y).
reach(X,Y) :- reach(X,Z), edge(Z,Y).
\end{verbatim}
Now we see what is implied by this set of rules: here we get
\{\verb|reach(a,b)|, \verb|reach(b,a)|, \verb|reach(a,a)|,
\verb|reach(b,b)|\} by taking the least fixed point of this program.
These are facts that must be true in any model consistent with the
partial knowledge we have of the \verb|edge/2| relation.  We have made
conservative assumptions and what is still true under these most
conservative assumptions must indeed be true.  We can say that this
program defines the ``definitely true'' atoms for \verb|reach|.

We have just seen how to conclude what \verb|reach| atoms must be
true; now how do we determine what atoms must be false?  We similarly
create a program using our assumptions of what \verb|edge| atoms are
true and false, but this time we assume that all \verb|edge| atoms
{\em not} known to be false are actually true.  So the program we get
is:
\begin{verbatim}
edge(a,b).
edge(b,a).
edge(c,b).
edge(c,a).
edge(a,a).
edge(b,b).
edge(c,c).
reach(X,Y) :- edge(X,Y).
reach(X,Y) :- reach(X,Z), edge(Z,Y).
\end{verbatim}
The seven \verb|edge| facts are those not known to be false,
i.e. those that might be true.  We use this program, taking the least
model, to see what \verb|reach| atoms ``might be true''.  We can say
this program defines the ``possibly true'' atoms.  In this case, we
get: \{\verb|reach(a,b)|, \verb|reach(b,a)|, \verb|reach(c,b)|,
\verb|reach(c,a)|, \verb|reach(a,a)|, \verb|reach(b,b)|,
\verb|reach(c,c)|\}.  Any \verb|reach| atom {\em not} in this set will
have to be false in any model consistent with our initial \verb|edge|
assumptions.  So we conclude that any \verb|reach| atom in the
complement of this set must be false in any consistent model.  So we
know that \verb|reach(b,c)| and \verb|reach(a,c)| must be false.  We
have made liberal assumptions about what might be true and anything
still not true under these most liberal assumptions must indeed be
false.

To review, we have described a way to use partial or incomplete
knowledge of facts that are used in the definitions of inductively
defined predicates to infer information about those predicates,
information including what instances must definitely be true and what
are possibly true (and whose complement must definitely be false.)
The idea for determining the definitely true defined instances is to
assume that all defining facts not known true are false.  Then
anything that must be true in this situation is definitely true.  To
determine the definitely false atoms, we assume all defining atoms not
known to be false are indeed true; anything that still isn't possibly
true under these assumptions must definitely be false.

We can use this idea to give a semantics to programs with negative
literals in their bodies; i.e., inductively defined predicates that
use negations in their definitions.  For simplicity we will consider
only propositional programs here.  In order to use a negative literal
to help infer a fact, we must know that the literal is true, i.e., the
atom that is negated is false.  But how can we determine such things?
We will use the idea we have just developed of reasoning with partial
information about open predicates to approach this problem.  We start
by initially disconnecting the negative literals in the bodies of
clauses from their positive forms and just thinking of them as new
propositions.  E.g., for the literal \verb|tnot(p)|\footnote{Recall
  that tnot is the tabled negation operator in XSB.}, we introduce a
new proposition symbol, say \verb|neg_p|.  We then replace all
negative literals in the program with their new positive forms to get
a purely positive program with open predicates.  The resulting program
can be seen as similar to the old program but now parameterized by the
newly introduced \verb|neg_*| predicates.

So let's apply our approach to reasoning about definitions with
partial knowledge about open predicates.  We begin by assuming that we
know nothing about these new open predicates, i.e., none of their
atoms are known true or false.  So to determine what defined
predicates must definitely be true, we interpret all the \verb|neg_*|
predicates as false, and see what is true in the resulting least
model.  Those defined atoms are now known true.  And to find the false
atoms, we interpret all the open predicates as true, take the least
model to see what could conceivably be true, and then take its
complement to find the definitely false atoms.

Let's consider an example; the initial general program with negation
is:
\begin{verbatim}
p :- r, tnot(t).
q :- r, tnot(s), tnot(u).
r :- s.
r.
s :- tnot(q), r.
t.
\end{verbatim}
We transform it to its open form:
\begin{verbatim}
p :- r, neg_t.
q :- r, neg_s, neg_u.
r :- s.
r.
s :- neg_q, r.
t.
\end{verbatim}
To determine the well-founded model of the original program, we will
maintain two programs with the open propositions: one that tells us
what atoms must be true (called the {\em definitely true} program),
and a program that tells us what atoms could possibly be true (called
the {\em possibly true} program) and therefore tells us what atoms
must be false.  Each program will be a conservative approximation and
we will iteratively modify the programs to improve their accuracy.

The initial definitely true program and definitely false program are:
\begin{verbatim}
%  definitely true               %  possibly true

p :- r, neg_t.                   p :- r, neg_t.
q :- r, neg_s, neg_u.            q :- r, neg_s, neg_u.
r :- s.                          r :- s.
r.                               r.
s :- neg_q, r.                   s :- neg_q, r.
t.                               t.
                                 neg_p.  neg_q.  neg_r. neg_s.  neg_t.  neg_u.
\end{verbatim}
In the definitely true program we assume that none of the \verb|neg_*|
propositions are true; in the possibly true program we assume they are
all true.  If we take the least model of the definitely true program,
we get \{\verb|r|, \verb|t|\}.  These propositions will true
regardless of the truth values of the \verb|neg_*| propositions, so we
will want them to be true in the well-founded model of the original
program.  And the least model of the possibly true program (on the
defined propositions) is \{\verb|p|, \verb|q|, \verb|r|, \verb|s|\}.
So any defined proposition symbol {\em not} in this set must be false,
and we will want it false in the well-founded model.

Now we can try to use each of these programs to improve the accuracy
of the other.  There really is a connection between the pair of
propositions, say \verb|p| and \verb|neg_p|: if one is true, then the
other should be false.  So if we deduce that \verb|r| is definitely
true in the well-founded model, then we know that \verb|neg_r| is
definitely false, i.e., not possibly true.  And that allows us to
update our possibly true program by deleting the fact for
\verb|neg_r|.  Similarly, if we know that \verb|u| is definitely false
then \verb|neg_u| is definitely true, and we can update our definitely
true program by adding \verb|neg_u| to it.  Having changed our
definitely true and possibly true programs, we can again find their
least fixed points and see if we have learned something new that will
allow us to further update the programs.  We continue to add
\verb|neg_*| atoms to the definitely true program and remove
\verb|neg_*| atoms from the possibly true program in this way until we
learn nothing new.  The resulting programs define the well-founded
semantics of the original program.

Consider this process for our example; the least model of the true
program contains r and t, so these are known true.  Since they are
known true, \verb|neg_r| and \verb|neg_t| must be false, so we can
remove them from the possibly true program, improving our estimate of
the possibly true atoms and getting an updated possibly true program:
\begin{verbatim}
%  definitely true               % possibly true

p :- r, neg_t.                   p :- r, neg_t.
q :- r, neg_s, neg_u.            q :- r, neg_s, neg_u.
r :- s.                          r :- s.
r.                               r.
s :- neg_q, r.                   s :- neg_q, r.
t.                               t.
                                 neg_p.  neg_q.  neg_s.  neg_u.
\end{verbatim}
Now looking at the least model of this new possibly true program, we
see that neither u nor p is possibly true, i.e., not in the least
model of the possibly true program and thus must be false.  So we can
improve our estimate of the true atoms by adding \verb|neg_u| and
\verb|neg_p| to our definitely true program, obtaining:
\begin{verbatim}
%  definitely true               % possibly true

p :- r, neg_t.                   p :- r, neg_t.
q :- r, neg_s, neg_u.            q :- r, neg_s, neg_u.
r :- s.                          r :- s.
r.                               r.
s :- neg_q, r.                   s :- neg_q, r.
t.                               t.
neg_p. neg_u.                    neg_p.  neg_q.  neg_s.  neg_u.
\end{verbatim}
Now looking at the current versions of the two programs: the neg
version of every defined atom in the least model of the definitely
true program has been removed from the possibly true program; and the
neg version of every defined atom not in the least model of the
possibly true program has been added to the definitely true program.
So we can no longer improve our estimates of the definitely true and
possibly true atoms, which leaves us with the well-founded model of
the original program.  The atoms true in the well-founded model are
the defined atoms in the least model of the final definitely true
program; the atoms false in the well-founded model are the defined
atoms {\em not} in the least model of the final possibly true program.
Thus for this program \verb|r| and \verb|t| are true in the
well-founded model, \verb|p| and \verb|u| are false, and \verb|s| and
\verb|q| are undefined.

To review this process: the definitely true program starts with no
\verb|neg_*| atoms and gains them as their positive counterparts are
found not to be possibly true.  The possibly true program starts with
all the \verb|neg_*| atoms and loses them as their positive
counterparts are found to be definitely true.  The process continues
until no more improvements can be made.  Then we read off the
well-founded model from the final programs.

